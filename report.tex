\documentclass[12pt]{article}

\usepackage{natbib}
\usepackage{xurl}
\usepackage{graphicx}

\title{Evolutionary rate of orthologs and paralogs}
\date{\today}
\author{Pavlos Pavlidis}

\begin{document}
\maketitle

\section*{Abstract}
abstract

\section*{Introduction}
Intro

\section*{Materials and Methods}
\subsection*{Data and filtering}
\paragraph{Data:} We downloaded the whole proteome of a set of organisms $S$ from the
Ensembl database~\citep{cunningham2022ensembl}, using custom scripts (Supplementary File XXXX). The
list of organisms is shown in Table~\ref{tab:listorgs}. Ensembl
proteomes are stored in the
\url{https://ftp.ensembl.org/pub/current_fasta/} directory of the
Ensembl ftp server and they are organized in separate folders based on
the scientific name of the organism (in a folder called `pep'). They
are represented in FASTA format with information-rich 
headers (i.e., the protein ID, gene ID, transcript ID as well as the
location of the protein in the genome is provided). This information
allowed us to filter sequences according to some predefined
criteria.

\paragraph{Filtering:} Prior to the analysis, we applied to filtering
procedures on the protein datasets. The first filter refers to
\emph{(i) Keep longer protein isoforms}. For 
each distinct Ensembl gene ID, we kept only the Protein ID that
corresponds to the longest polypeptide sequence. The second filtering
procedure refers to
\emph{(ii) keep proteins with a minimum length}. As shown in
Table~\ref{tab:lengthdist}, a protein length (after applying filter
(i)), ranges between less than ten and several thousands of amino acids. We
kept only proteins comprise a minimum length of 100 amino acids
since this value corresponds to approximate the 5\% of protein
lengths (Table~\ref{tab:lengthdist}).

\begin{table}[htbp!]
  \caption{The percentiles of protein lengths for the organisms used
    in the study}
  \label{tab:lengthdist}
  \begin{tabular}{|l|lllllll|}
    
    & 0                        & 5  & 10  & 50  & 90   & 95   & 100
    \\
    \hline
    {\it Canis lupus familiaris} & 15 & 100 & 134 & 410  & 1077 & 1440  & 27097 \\
    {\it Equus caballus}          & 13 & 110 & 154 & 425  & 1105 & 1452  & 34311 \\
    {\it Felis catus}             & 13 & 105 & 147 & 425  & 1096 & 1461  & 27108 \\
    {\it Homo sapiens}            & 2  & 107 & 137 & 410  & 1066 & 1455  & 35991 \\
    {\it Macaca mulatta}          & 17 & 106 & 126 & 409  & 1084 & 1419  & 35478 \\
    {\it Mus musculus}            & 3  & 112 & 143 & 384  & 1033 & 1401  & 35390 \\
    {\it Pan troglodytes} & 18      & 90 & 120 & 384 & 1035 & 1399 & 34270\\
    {\it Pongo abelii}            & 4  & 102 & 136 & 411  & 1068 & 1430  & 34347 \\
    {\it Sciurus vulgaris}        & 18 & 89  & 119 & 359  & 983  & 1315  &
    34292\\
    
  \end{tabular}
\end{table}

\section*{Results}

\bibliographystyle{plainnat}
\bibliography{biblio}

\end{document}
